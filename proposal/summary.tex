\documentclass[12pt]{article}
\usepackage{fullpage,enumitem,amsmath,amssymb,graphicx}
\usepackage{hyperref}
\usepackage{physics}
\usepackage{minted}

\begin{document}

\section{Project Description}

\paragraph{Goal.}
The objective of this project is to generate bash commands from natural
language using a deep neural network. Novitiates often find the terminal
interface perplexing and are quickly overwhelmed by the syntax of bash
commands. Even experienced engineers frequently consult man-pages, online
documentation, and online forms like stackoverflow to learn about the
particulars of various commands. This project aims to ease that burden on new
and experienced users alike. An interesting minimum viable product in this
space in this space is ``aish,'' short for AI-shell, which is part of the
\href{http://anix.org}{AInix} Kernel project. In the future, we hope that this
field of research will be able to construct sequences of bash commands,
automating entire workflows from simple natural language descriptions. However,
for this project, we focus solely on executing atomic commands.
\par
This project is a variant of semantic parsing. In the preceding paper summary,
we discussed the Seq2SQL model, which is used to translate natural language to
SQL commands. There are several novel techniques used in this paper that are
applicable to our project. First, we note that they prune the total number of
acceptable translations using the structure of the input and the SQL grammar.
Using ``augmented pointer networks'', the input includes all the tokens
required to generate SQL come from the input, including column names, SQL
keywords, and the tokenized question. We would like to try something similar in
our translator.
\par
The paper also introduces an element of reinforcement learning. Because some
components of the SQL query are order agnostic, i.e. where clause order, they
cannot simply compare the generated query to the golden query. The reward
assigned to a given query depends on (1) whether it is a valid SQL query and
(2) whether it executes to the correct result. This reward is factored in with
the other loss calculations to calculate total loss on a translation. While we
would like to do something similar, we have a few issues. First, most of the
bash commands executed in a naked environment will probably fail, even if
properly formed, due to missing paths. Second, there is no easy way to tell
whether a bash command is well-formed. While most commands could be expressed
as a context-free grammar, generating those from help pages would prove
prohibitive. In sum, if we incorporate policy generation, we will need to
define a new loss function, perhaps defined on the number of matches between
sets of flags.

\paragraph{Task.}
The task here is to correctly translate natural language commands to bash
commands. For example, one might query this system with, ``How do I unpack a
tarball?'' To which the answer would be ``tar -xvzf $<$placeholder$>$''. To
start, we will attempt to match the golden answer exactly, and measure accuracy
over the test set.

\paragraph{Data.}
We will be using the dataset provided by ``The NLC2CMD Challenge'' hosted
\href{https://nlc2cmd.us-east.mybluemix.net/}{here}. This was a competition
held in 2020 by NeurIPS. The
\href{https://github.com/IBM/clai/blob/nlc2cmd/docs/nl2bash-data.md}{dataset}
consists of 10,000 parallel translations of English and bash. The main
\href{https://github.com/IBM/clai/tree/nlc2cmd}{github page} for the challenge
provides several tools for loading and preprocessing the data. However, we plan
to try several different tokenization techniques, guided by our model
selections, BART and T5. We also plan to augment the dataset by generating text
for placeholders, i.e. different paths, for eligible commands.

\paragraph{Methods.}
We plan on fine-tuning a BART transformer for our model. In addition to tuning
the model for bash commands, we would like to experiment with the augmented
pointers in the input for the BART model. Time permitting, we would also like
to introduce an element reinforcement learning by defining a reward function
over flag permutations and factoring that into our loss.

\paragraph{Baselines.}
The competition provides several baselines. The primary baseline is a model
called Tellina, which achieves 13.8\% accuracy on the dataset. The competition
winner, Magnum, used OpenNMT, a holistic python framework for neural machine
translation, and achieved an accuracy score of 53.2\%. While we hope to improve
on the best model's performance, we suspect that our performance will lie
somewhere in between these two.

\paragraph{Evaluation.}
The competition clearly defines a metric for evaluation, which we will use. The
data consists of natural language to bash command pairs, . Our model, $A$,
will implement a top 5 translator from a natural language command, $nlc$, to a
prediction, $c$, and confidence, $\delta$, tuple.
\[
	\begin{aligned}
		G(nlc)
		 & =\{C\mid \text{Bash command that achieves the task described in} nlc\} \\
		A : nlc
		 & \rightarrow \{p \mid p = \langle c, \delta\rangle\}                    \\
		|A(nlc)|
		 & \le 5                                                                  \\
	\end{aligned}
\]
The normalized score of a prediction, $p=\langle c, \delta\rangle$ is:
\[
	\begin{aligned}
		S(p)
		 & = \max_{C\in G(nlc)}\sum_{i\in[1,T]}\frac{\delta}{T}\times\left(
		\mathbb{I}[U(c)_i=U(C)i]\times\frac{1}{2}\left(
			1+\frac{1}{N}\left(X\right)\right) -\mathbb{I}[U(c)_i\ne U(C)_i]
		\right)                                                                    \\
		\text{where}
		 &                                                                         \\
		X
		 & = 2\times \abs{F(U(c)_i)\cap F(U(C)_i)} - \abs{F(U(c)_i)\cup F(U(C)_i)} \\
		U(c)
		 & = \text{sequence of Bash utilities in a command $c$}                    \\
		F(u)
		 & = \text{the set of flags for an utility $u$}                            \\
		T
		 & = \max\left(\abs{U(c)}, \abs{U(C)}\right)                               \\
		N
		 & = \max\left(\abs{F(U(c)_i)}, \abs{F(U(C)_i)}\right)                     \\
	\end{aligned}
\]
The overall score is given by the following:
\[
	\begin{aligned}
		Score(A(nlc))
		 & =
		\begin{cases}
			\max_{p\in A(nlc)}S(p),
			 & \text{ if }\exists_{p\in A(nlc)}\text{ such that }S(p) > 0 \\
			\frac{1}{\abs{A(nlc)}}\sum_{p\in A(nlc)}S(p),
			 & \text{otherwise}                                           \\
		\end{cases}
	\end{aligned}
\]
This scoring mechanism promotes precision and recall of the correct binary and
flags, weighted by confidence. There are more caveats and details under the
Task section of the challenge website.

\end{document}
