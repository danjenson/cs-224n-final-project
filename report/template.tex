\documentclass{article}

\usepackage[final]{neurips_2019}

\usepackage[utf8]{inputenc}
\usepackage[T1]{fontenc}
\usepackage{url}
\usepackage{booktabs}
\usepackage{amsfonts}
\usepackage{nicefrac}
\usepackage{microtype}
\usepackage{graphicx}
\usepackage{xcolor}
\usepackage{xcolor}
\usepackage[colorlinks = true,
            linkcolor = blue,
            urlcolor  = blue,
            citecolor = blue,
            anchorcolor = blue]{hyperref}

\newcommand{\note}[1]{\textcolor{blue}{{#1}}}

\title{
  Title of your project \\
  \vspace{1em}
  \small{\normalfont Stanford CS224N \{Custom, Default\} Project}  % Select one and delete the other
}

\author{
  Name \\
  Department of Computer Science \\
  Stanford University \\
  \texttt{name@stanford.edu} \\
  % Examples of more authors
%   \And
%   Name \\
%   Department of Computer Science \\
%   Stanford University \\
%   \texttt{name@stanford.edu} \\
%   \And
%   Name \\
%   Department of Computer Science \\
%   Stanford University \\
%   \texttt{name@stanford.edu}
}

\begin{document}

\maketitle

\begin{abstract}
	An abstract should concisely (less than 300 words) motivate the problem, describe your aims, describe your contribution, and highlight your main finding(s).
\end{abstract}


\section{Key Information to include}
\begin{itemize}
	\item Mentor:
	\item External Collaborators (if you have any):
	\item Sharing project:
\end{itemize}

% {\color{red} This template does not contain the full instruction set for this assignment; please refer back to the milestone instructions PDF.}

\section{Introduction}
The introduction explains the problem, why it's difficult, interesting, or important, how and why current methods succeed/fail at the problem, and explains the key ideas of your approach and results. Though an introduction covers similar material as an abstract, the introduction gives more space for motivation, detail, references to existing work, and to capture the reader's interest.

\section{Related Work}
This section helps the reader understand the research context of your work, by providing an overview of existing work in the area.

\color{red}
\section{Approach}
\color{black}
The general approach consisted of two principle methods: (1) text generation
and (2) translation. First, it is important to note that Bash is not a
context-free grammar. It admits of very little recursion and, while most
binaries are POSIX compliant, their interfaces are non-standardized. Flags
often carry different semantic meaning and imply different tasks when employed
by different binaries. Moreover, flags often override, modify, or cancel the
intent of other flags in the same command, introducing complex dependencies.
These dependencies can also shift as the order of the flags and their arguments
are permuted. In sum, the meaning of a flag is almost entirely provided by the
invoking binary and its location in the sequence of arguments. This introduces
difficulties in fine tuning embeddings, since training may attempt to encode
vastly different meanings in the same embedding, although this is not always a
problem(TODO cite paper on deconstructing embeddings). Given sufficient
training data, it is likely that that the models may eventually learn correct
contextual meaning when employed by different binaries, but we found 10,000
rows insufficient for the task. This line of thinking inspired our first
approach, text generation using GPT2.
\par
While at first this task appears to be a straightforward translation task,
after considering bash more closely, one can see that it does not admit of many
properties or structures of natural language. Accordingly, we thought that
rather than trying to properly translate natural language into bash, we could
train a model to hallucinate bash ``stories'' given natural language. The
high-level idea here is that we finetune a GPT2 model, showing it complete
stories that consist of both a natural language portion and a bash portion with
some added separating tokens. While training, GPT2 learns common storylines,
and at testing time, we feed the trained model only the first half of the
story, i.e. the natural language portion, and ask it to complete the story,
hoping that it will generate bash commands as the most likely story completion.
In many respects, this idea performs quite well; however, a significant issue
with this approach is constraining responses from GPT2. How long should the
story be? When does the real content of the ``bash story'' start and end? What
happens when GPT2 has multiple endings? These questions are detailed in the
error analysis section.
\par
The second approach we used was a more traditional seq2seq language modeling
approach. Pre-trained models for BART and T5 are easily fine-tuned for
translation tasks. The requirement from the course website: This section
details your approach(es) to the problem. For example, this is where you would
describe the architecture of your neural network(s) or your novel algorithms.
For projects that aren’t describing a novel methodology, this section might
describe in more detail whatever technical tools (e.g. probing methods, proof
techniques, interpretability algorithms, etc.) you’ll use in your experiments.

Yingxiao's thoughts: here, we can talk about the three different models we tried, different tokenziers, different trainers, different post processing (we can also talk about this later), and anything in our codes worth mentioning

\section{Experiments}
This section contains the following.

\subsection{Data}
Describe the dataset(s) you are using (provide references). If it's not already clear, make sure the associated task is clearly described.
Being precise about the exact form of the input and output can be very useful for readers attempting to understand your work, especially if you've defined your own task.

\subsection{Evaluation method}
Describe the evaluation metric(s) you use, plus any other details necessary to understand your evaluation.
Some projects will have clear metrics from prior work on given datasets, but we realize that other projects will define their own metrics.
If you're defining your own metrics, be clear as to what you're hoping to measure with each evaluation method (whether quantitative or qualitative, automatic or human-defined!), and how it's defined.

\color{red}
\subsection{Experimental details (Dan's part)}
\color{black}
For this task, we tested 3 models: HuggingFace's
\href{https://huggingface.co/gpt2}{GPT2}\cite{gpt2},
\href{https://huggingface.co/facebook/bart-large}{Facebook's BART
Large}\cite{bart}, and \href{https://huggingface.co/t5-large}{Google's T5
Large}. GPT2 was a causal model, predicting text from context, while the other
two were traditional seq2seq models. Each was trained for 5, 10, and 25 epochs.
Batch size was limited to 10 examples, except for T5, which had to be reduced
to 5. For training, we used the AdamW optimizer with weight decay
regularization. The learning rate was linear with a warmup of 100 steps. The
dataset was split into 98\% training and 2\% test sets. Given that the dataset
was approximately 10,000 parallel translations, the test set was slightly over
200 examples. Training time for GPT, BART, and T5 was approximately 1, 1.5, and 4.5
hours, respectively, on an Azure NC6 instance with a Tesla V100 PCIe
16GB GPU. All 3 models used cross-entropy loss for training, but were scored on
the test set using the NLC2CMD metric at the end of each epoch.

\color{red}
\subsection{Results (Dan's part)}
\color{black}
The requirement from the course website: Report the quantitative results that you have found so far. Use a table or plot to compare results and compare against baselines.
Comment on your quantitative results. Are they what you expected? Better than you expected? Worse than you expected? Why do you think that is? What does that tell you about your approach?

Yingxiao's thought: show them the loss-epoch curve, the metric-epoch curve, and one table summarizing the best scores for the gpt2, bart, t5, and baseline gpt3, with different postprocessing functions. Please comment as much as you could in this section. Mention that our results became better and better after more epochs, although the score did not improve. Any thought on the causual / seq2seq models?


\section{Analysis}
Your report should include \textit{qualitative evaluation}. That is, try to understand your system (e.g. how it works, when it succeeds and when it fails) by inspecting key characteristics or outputs of your model.

\section{Conclusion}
Summarize the main findings of your project, and what you have learnt. Highlight your achievements, and note the primary limitations of your work. If you like, you can describe avenues for future work.


\bibliographystyle{unsrt}
\bibliography{references}


\appendix

\section{Appendix (optional)}
If you wish, you can include an appendix, which should be part of the main PDF, and does not count towards the 6-8 page limit.
Appendices can be useful to supply extra details, examples, figures, results, visualizations, etc., that you couldn't fit into the main paper. However, your grader \textit{does not} have to read your appendix, and you should assume that you will be graded based on the content of the main part of your paper only.

\end{document}
