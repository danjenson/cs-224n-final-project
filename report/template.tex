\documentclass{article}

\usepackage[final]{neurips_2019}

\usepackage[utf8]{inputenc}
\usepackage[T1]{fontenc}
\usepackage{hyperref}
\usepackage{url}
\usepackage{booktabs}
\usepackage{amsfonts}
\usepackage{nicefrac}
\usepackage{microtype}
\usepackage{graphicx}
\usepackage{xcolor}
\usepackage{lipsum}

\newcommand{\note}[1]{\textcolor{blue}{{#1}}}

\title{
  Title of your project \\
  \vspace{1em}
  \small{\normalfont Stanford CS224N \{Custom, Default\} Project}  % Select one and delete the other
}

\author{
  Name \\
  Department of Computer Science \\
  Stanford University \\
  \texttt{name@stanford.edu} \\
  % Examples of more authors
%   \And
%   Name \\
%   Department of Computer Science \\
%   Stanford University \\
%   \texttt{name@stanford.edu} \\
%   \And
%   Name \\
%   Department of Computer Science \\
%   Stanford University \\
%   \texttt{name@stanford.edu}
}

\begin{document}

\maketitle

\begin{abstract}
An abstract should concisely (less than 300 words) motivate the problem, describe your aims, describe your contribution, and highlight your main finding(s). 
\end{abstract}


\section{Key Information to include}
\begin{itemize}
    \item Mentor:
    \item External Collaborators (if you have any):
    \item Sharing project:
\end{itemize}

% {\color{red} This template does not contain the full instruction set for this assignment; please refer back to the milestone instructions PDF.}

\section{Introduction}
The introduction explains the problem, why it's difficult, interesting, or important, how and why current methods succeed/fail at the problem, and explains the key ideas of your approach and results. Though an introduction covers similar material as an abstract, the introduction gives more space for motivation, detail, references to existing work, and to capture the reader's interest.

\section{Related Work}
This section helps the reader understand the research context of your work, by providing an overview of existing work in the area.

\color{red}
\section{Approach (Dan's part)}
\color{black}
The requirement from the course website: This section details your approach(es) to the problem. For example, this is where you would describe the architecture of your neural network(s) or your novel algorithms. For projects that aren’t describing a novel methodology, this section might describe in more detail whatever technical tools (e.g. probing methods, proof techniques, interpretability algorithms, etc.) you’ll use
in your experiments.

Yingxiao's thoughts: here, we can talk about the three different models we tried, different tokenziers, different trainers, different post processing (we can also talk about this later), and anything in our codes worth mentioning

\section{Experiments}
This section contains the following.

\subsection{Data}
Describe the dataset(s) you are using (provide references). If it's not already clear, make sure the associated task is clearly described.
Being precise about the exact form of the input and output can be very useful for readers attempting to understand your work, especially if you've defined your own task.

\subsection{Evaluation method}
Describe the evaluation metric(s) you use, plus any other details necessary to understand your evaluation.
Some projects will have clear metrics from prior work on given datasets, but we realize that other projects will define their own metrics.
If you're defining your own metrics, be clear as to what you're hoping to measure with each evaluation method (whether quantitative or qualitative, automatic or human-defined!), and how it's defined.

\color{red}
\subsection{Experimental details (Dan's part)}
\color{black}
The requirement from the course website: Report how you ran your experiments (e.g. model configurations, learning rate, training time, etc.)

Yingxiao's thought: interpret the configuration we used for training

\color{red}
\subsection{Results (Dan's part)}
\color{black}
The requirement from the course website: Report the quantitative results that you have found so far. Use a table or plot to compare results and compare against baselines.
Comment on your quantitative results. Are they what you expected? Better than you expected? Worse than you expected? Why do you think that is? What does that tell you about your approach?

Yingxiao's thought: show them the loss-epoch curve, the metric-epoch curve, and one table summarizing the best scores for the gpt2, bart, t5, and baseline gpt3, with different postprocessing functions. Please comment as much as you could in this section. Mention that our results became better and better after more epochs, although the score did not improve. Any thought on the causual / seq2seq models?


\section{Analysis}
Your report should include \textit{qualitative evaluation}. That is, try to understand your system (e.g. how it works, when it succeeds and when it fails) by inspecting key characteristics or outputs of your model.

\section{Conclusion}
Summarize the main findings of your project, and what you have learnt. Highlight your achievements, and note the primary limitations of your work. If you like, you can describe avenues for future work.


\bibliographystyle{unsrt}
\bibliography{references}

\appendix

\section{Appendix (optional)}
If you wish, you can include an appendix, which should be part of the main PDF, and does not count towards the 6-8 page limit.
Appendices can be useful to supply extra details, examples, figures, results, visualizations, etc., that you couldn't fit into the main paper. However, your grader \textit{does not} have to read your appendix, and you should assume that you will be graded based on the content of the main part of your paper only.

\end{document}
