\subsection{Data}
The dataset we used is from the
``\href{https://nlc2cmd.us-east.mybluemix.net/}{The NLC2CMD Competition},''
consisting of 10,000 parallel translations of English (labelled
``invocation'') and Bash commands (labelled ``cmd''). Here is an example:
\begin{verbatim}
invocation: Assign permissions 755 to directories in the current directory tree
cmd: find . -type d -print0 | xargs -0 chmod 755
\end{verbatim}
Most of the invocations in the dataset involve a sequence of different tasks,
and consequently the Bash commands often consist of a series of pipelines. In
addition, since the Bash commands contain identifiers, such as directory paths,
file names, and permissions, a templatization scheme has been imposed by
converting shell commands into their corresponding abstract syntax trees
(ASTs), replacing identifier nodes with placeholders, and then recombining the
command. Applying this process to the previous command produces the following
templated command:
\begin{verbatim}
templated cmd: find Path -type d -print0 | xargs -0 -I chmod Permission
\end{verbatim}
This helps the model to generalize during training, without getting distracted
by a myriad of specific identifiers.
\par
The dataset was split into the training and test sets with a ratio of 0.98 to 0.02, yielding 10,140 training examples and 207 test examples. The invocations and templated commands must be tokenized by the same tokenizer used in training each model; accordingly, the outputs vary by model and tokenizer. For instance, the BART tokenizer yields the following encoded example:
\begin{verbatim}
<s>Assign permissions 755 to directories in the current directory 
tree</s></s>find Path -type d -print0 | xargs -0 -I chmod Permission</s>
\end{verbatim}
While the T5 tokenizer yields the following:
\begin{verbatim}
Assign permissions 755 to directories in the current directory 
tree</s> find Path -type d -print0 | xargs -0 -I chmod Permission</s>
\end{verbatim}
The GPT-2 model does not ingest input as pairs, but instead as entire sections of
text. Natively, it only defines the beginning and ending special tokens, so
we had to develop our own encoding scheme to communicate the structure of our
input to the GPT-2 model. The primary objective of the encoding scheme is to
introduce tokens that signal the beginning of natural language and Bash commands. For this, we used the \texttt{<|source|>} and \texttt{<|target|>} tokens, but these could have been any tokens unlikely to be used by Bash utilities or arguments. The general template for the encoding scheme was as follows:
\begin{verbatim}
<bos_token> <source_token> <invocation> <target_token>
                                              <templated cmd> <eos_token> 
\end{verbatim}
Using the above example, this scheme produces the following encoding:
\begin{verbatim}
<|endoftext|> <|source|> Assign permissions 755 to directories in the current
directory tree <|target|> find Path -type d -print0 | xargs -0 -I chmod 
Permission <|endoftext|>
\end{verbatim}
While this encoding was sufficient for training, we still found it difficult
for GPT-2 to learn the semantics of our special tokens.


\subsection{Evaluation method}
The standard cross-entropy loss function was used to train the models. But a
more robust metric measuring the accuracy of the model predictions defined by
the competition was used to evaluate the performance of our models. The metric
is expressed mathematically:
\begin{align*}
	S(p) & =\sum_{i\in[1,T]}\frac{1}{T}\times\left(
	\mathbb{I}[U(c)_i=U(C)i]\times\frac{1}{2}\left(
		1+\frac{1}{N}\left(X\right)\right) -\mathbb{I}[U(c)_i\ne U(C)_i]
	\right)
\end{align*}
$U(x)$ is a sequence of Bash binaries in a command $x$, $c$ is the
predicted Bash command and $C$ is the ground truth Bash command. Apart from
measuring whether the executables in the two commands match, an additional
variable $X$ has been introduced to measure whether the flags associated with
each utility match or not:
\begin{equation*}
	X = 2\times
	|F(U(c)_i)\cap F(U(C)_i)| - |F(U(c)_i)\cup F(U(C)_i)|
\end{equation*}
$F(x)$ refers to the set of Bash flags in a command $x$. $T$ is the
maximum length between $U(c)$ and $U(C)$ while $N$ is the maximum size between
$F(c)$ and $F(C)$. Since the order of flags does not matter, these are set
operations.
\par
It is important to note that this metric is extremely strict, assigning a score
of nearly -1.0 for predicting the incorrect binary alone. It also penalizes
extra and incorrect flags and arguments. The return value ranges from -1.0 to
1.0, and a score of 1.0 is only awarded when a command is precisely equivalent
to the golden one.
