\documentclass{article}

\usepackage[final]{neurips_2019}

\usepackage[utf8]{inputenc}
\usepackage[T1]{fontenc}
\usepackage{hyperref}
\usepackage{url}
\usepackage{booktabs}
\usepackage{amsfonts}
\usepackage{nicefrac}
\usepackage{microtype}
\usepackage{graphicx}
\usepackage{verbatim}
\usepackage{xcolor}
\usepackage{lipsum}
\usepackage{fullpage,enumitem,amsmath,amssymb,graphicx}
\newcommand{\note}[1]{\textcolor{blue}{{#1}}}

\title{
  Milestone Report: Translating Natural Language to Bash Commands using Deep Neural Network \\
  \vspace{1em}
  \small{\normalfont Stanford CS224N Custom Project  }
}

\author{
 Daniel Jenson \\
  Department of ??? \\
  Stanford University \\
  \texttt{djenson@stanford.edu} \\
  % Examples of more authors
  \And
  Yingxiao Liu \\
  Department of Civil and Environmental Engineering \\
  Stanford University \\
  \texttt{liuyx@stanford.edu} \\
  % Examples of more authors
%   \And
%   Name \\
%   Department of Computer Science \\
%   Stanford University \\
%   \texttt{name@stanford.edu} \\
%   \And
%   Name \\
%   Department of Computer Science \\
%   Stanford University \\
%   \texttt{name@stanford.edu}
}

\begin{document}

\maketitle

\begin{abstract}
  Your abstract should motivate the problem, describe your goals, and highlight your main findings. Given that your project is still in progress, it is okay if your findings are what you are still working on.
\end{abstract}


\section{Key Information to include}
\begin{itemize}
    \item TA mentor: Ethan A. Chi
    \item External collaborators: No
    \item External mentor: No
    \item Sharing project: No
\end{itemize}

% {\color{red} This template does not contain the full instruction set for this assignment; please refer back to the milestone instructions PDF.}

\section{Approach}
This section details your approach to the problem. 
\begin{itemize}
    \item Please be specific when describing your main approaches. You may want to include key equations and figures (though it is fine if you want to defer creating time-consuming figures until the final report).
    \item Describe your baselines. Depending on space constraints and how standard your baseline is, you might do this in detail or simply refer to other papers for details. Default project teams can do the latter when describing the provided baseline model.
    \item If any part of your approach is original, make it clear. For models and techniques that are not yours, provide references.
    \item If you are using any code that you did not write yourself, make it clear and provide a reference or link. 
    When describing something you coded yourself, make it clear.
\end{itemize} 


\section{Experiments}
\subsection{Dataset}
The dataset we used is from the ``The NLC2CMD Challenge'' hosted
\href{https://nlc2cmd.us-east.mybluemix.net/}{here}, consists of 10,000 parallel translations of English and bash. One example is like
\begin{verbatim}
ENG: Assign permissions 755 to directories in the current directory tree
CMD: find . -type d -print0 | xargs -0 chmod 755
\end{verbatim}
Since the bash examples consist specific directory paths, file names and permissions, we followed the guidance \href{https://github.com/IBM/clai/tree/nlc2cmd}{here} to parse the bash command into the corresponding template form as
\begin{verbatim}
CMD: find Path -type d -print0 | xargs -0 -I chmod Permission
\end{verbatim}
both to better capture the command structure and for simplicity. Finally, we grouped the English and the bash template together into the form
\begin{verbatim}
<eos_token> <ENG_token> <English> <CMD_token> <CMD_template> <eos_token> 
\end{verbatim}
by inserting tokens, which can be readily fed into the tokenizer provided by Huggingface. The dataset was then split into the training, validation and test sets with a ratio of 0.96 : 0.02 : 0.02.

\subsection{Evaluation method}
During training, the 
\subsection{Experimental Details}
\subsection{Results}


\section{Future work}
Describe what you plan to do for the rest of the project and why.


\bibliographystyle{unsrt}
\bibliography{references}

\end{document}
