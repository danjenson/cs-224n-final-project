\documentclass{article}

\usepackage[final]{neurips_2019}

\usepackage[utf8]{inputenc}
\usepackage[T1]{fontenc}
\usepackage{hyperref}
\usepackage{url}
\usepackage{booktabs}
\usepackage{amsfonts}
\usepackage{nicefrac}
\usepackage{microtype}
\usepackage{graphicx}
\usepackage{verbatim}
\usepackage{xcolor}
\usepackage{lipsum}
\usepackage{fullpage,enumitem,amsmath,amssymb,graphicx}
\newcommand{\note}[1]{\textcolor{blue}{{#1}}}
\hypersetup{colorlinks}
\title{
  Milestone Report: Translating Natural Language to Bash Commands using Deep Neural Networks \\
  \vspace{1em}
  \small{\normalfont Stanford CS224N Custom Project  }
}

\author{
 Daniel Jenson \\
  Department of Management Science \& Engineering \\
  Stanford University \\
  \texttt{djenson@stanford.edu} \\
  % Examples of more authors
  \And
  Yingxiao Liu \\
  Department of Civil and Environmental Engineering \\
  Stanford University \\
  \texttt{liuyx@stanford.edu} \\
  % Examples of more authors
%   \And
%   Name \\
%   Department of Computer Science \\
%   Stanford University \\
%   \texttt{name@stanford.edu} \\
%   \And
%   Name \\
%   Department of Computer Science \\
%   Stanford University \\
%   \texttt{name@stanford.edu}
}

\begin{document}

\maketitle

\begin{abstract}
	The objective of this project is to generate bash commands from natural
	language using a deep neural network. We experimented with several models,
	including GPT2, BART, and T5, as well as different tokenization schemes to
	improve model performance on the NLC2CMD dataset. We found that T5,
	specifically v1.1 released by Google, performs the best on this task.
\end{abstract}


\section{Key Information to include}
\begin{itemize}
	\item TA mentor: Ethan A. Chi
	\item External collaborators: No
	\item External mentor: No
	\item Sharing project: No
\end{itemize}

% {\color{red} This template does not contain the full instruction set for this assignment; please refer back to the milestone instructions PDF.}

\section{Approach}
The NLC2CMD Challenge was held only once at NeurIPS in 2020 and competing teams
put out working papers at best, but most often simply scattered notes across
Github. The dataset is simple a JSON consisting of translation pairs. Because
of this, a great deal of work had to go into preprocessing and encoding the
data, as well as becoming familiar with the HuggingFace infrastructure. We used
several trivial encodings to start, but found only one that achieves decent
performance, detailed in the dataset section. We found that the effectiveness
of the encoding varies by model and training objective, so this will have to be
tuned as we experiment with different models.
\par
The competition also provided several utilities of which we availed ourselves.
First, they released a bash to AST parser, which can also write an AST back to
a templated form, e.g. replacing file paths with ``File'' and regular
expressions with the token ``Regex.'' This allows the model to learn
placeholder values for commands. They also provided a function to compute their
competition scoring metric, which we use to establish our baseline.
\par
We used GPT-2 as our first base model, as most top competitors used this model
and it appears to provide a solid foundation for causal language modeling. We
also used the corresponding HuggingFace tokenizer. We experimented with BART,
but this model requires more work, as repurposing it for casual language
modeling has proven difficult; many original model weights are left
uninitialized, since this was not the original training objective of the model.
Our GPT-2 model achieved a performance of -0.21 using the evaluation criteria
provided by the competition, which is just shy of the -0.19 achieved by the
GPT-3 baseline.
\par

\section{Experiments}
\subsection{Dataset}
The dataset we used is from the ``\href{https://nlc2cmd.us-east.mybluemix.net/}{The NLC2CMD Competition},'' consisting of 10,000 parallel translations of English and bash, such as the following:
\begin{verbatim}
invocation: Assign permissions 755 to directories in the current directory tree
cmd: find . -type d -print0 | xargs -0 chmod 755
\end{verbatim}
Because the bash commands contain identifiers, such as directory paths, file names, and permissions, we followed the guidance \href{https://github.com/IBM/clai/tree/nlc2cmd}{here} to translate bash commands, such as the one above, into the corresponding template form:
\begin{verbatim}
cmd: find Path -type d -print0 | xargs -0 -I chmod Permission
\end{verbatim}
Finally, we combined the natural language and bash command pair into a single recording, encoding it with the following format:
\begin{verbatim}
<eos> <nl> <natural language command> <cmd> <bash command> <eos> 
\end{verbatim}
For training, we created three datasets, training, validation, and test, with
96\%, 2\%, and 2\% of the data, respectively.

\subsection{Evaluation method}
We used the cross entropy loss to train the model, but to measure model
performance we used the metric defined by the competition. The metric permits
submission of up to five translations for each natural language command.
However, as we have not yet implemented beam search, we submit only one
translation with a confidence of 1.0. The metric is computed as follows:
\begin{align*}
	S(p) & =\sum_{i\in[1,T]}\frac{1}{T}\times\left(
	\mathbb{I}[U(c)_i=U(C)i]\times\frac{1}{2}\left(
		1+\frac{1}{N}\left(X\right)\right) -\mathbb{I}[U(c)_i\ne U(C)_i]
	\right)
\end{align*}
where $U(x)$ is a sequence of bash utilities in a command $x$, $c$ is the
predicted bash command and $C$ is the ground truth bash command. $X = 2\times
	|F(U(c)_i)\cap F(U(C)_i)| - |F(U(c)_i)\cup F(U(C)_i)|$ where $F(x)$ refers to
the set of bash flags in a command $x$. $T$ is the maximum length between
$U(c)$ and $U(C)$ while $N$ is the maximum size between $F(c)$ and $F(C)$. This
is a very strict metric penalizing for both incorrect and extra bash
utilities and flags.
\subsection{Experimental Details}
We fine tuned a GPT-2 model from the Huggingface AutoModelForCausalLM
pretrained models on our dataset. We trained in 5 epochs with a batch size of
50; we found that further training led to overfitting on the training set and
worse performance on the validation set. We used the AdamW optimizer with an
initial learning rate of 5e-5.

\subsection{Results}
The change of training loss during training has been plotted in the Figure 1
below. Initially, the loss decreased rapidly but after 3 epochs it became more
or less stable around 1.4.

\begin{figure}[ht!]
	\centering
	\includegraphics[width = 160px]{training_loss.png}
	\caption{Cross entropy loss by epoch.}
	\label{overfitting}
\end{figure}
\par
Using the evaluation metric defined previously, without any post processing of
the prediction, the model achieves a score of only -0.6 (a dummy model will get
a score of -1.0 while an oracle will score 1.0). After conducting an error
analysis, we found that the model tended to output repeated sequences or
redundant pipelines, which were heavily penalized by the metric. Therefore, a
post processing function has been added to remove adjacent repeated words, and
limit the maximum number of pipelines to be 3. Then, our model achieves a score
of -0.21538. For comparison, the baseline model using GPT-3 scored -0.19.
\par
While the score appears quite low, looking at the predictions, we can see that
the model is making substantively correct predictions. In most cases, the
prediction is very close to the ground truth:
\begin{verbatim}
Prediction: find Path -nouser -exec rm {} \;
Truth     : find Path -nouser -ok rm {} \; 
\end{verbatim}
However, the model has a tendency to ramble, adding additional, unnecessary command sequences:
\begin{verbatim}
Prediction: cat File | sort -n -r | grep -v Regex
Truth     : cat File | sort -r -h 
\end{verbatim}
Another example with repeated sequences:
\begin{verbatim}
Prediction: mount Regex -o remount,rw Regex mount Regex -o remount,rw 
            Regex mount Regex -o remount ...
Truth     : mount -o remount,ro -t yaffs2 Regex Regex 
\end{verbatim}
Fixing these more minor mistakes will significantly increase the performance.


\section{Future work}
In future work, we will create a much larger dataset, as well as experiment
with different command line read-evaluate-print-loops (REPL). Outside of bash
AST parsing, this framework is REPL-agnostic and could be used for any
collection of structured commands. We will experiment with Nushell, a
recent shell written in Rust, which has much cleaner syntax than POSIX
compliant shells. Lastly, we will incorporate input selection for identifiers
to avoid using templated commands, similar to that used by Zhong, et
al\cite{zhong2017seq2sql}.


\bibliographystyle{unsrt}
\bibliography{references}

\end{document}
